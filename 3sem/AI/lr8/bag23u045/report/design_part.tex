\chapter{Конструкторская часть}

Для анализа близости используется вектор документа, способ получения которого описывался в лабораторной работе №7. 

\section{Подготовка данных}

Для метода главных компонент вектора дополняются словами из других векторов (их количество равно 0 в данном векторе).
Это делается для соблюдения единой размерность векторов.

\section{Метрики}

Косинусоидальное сходство вычисляет сходство двух векторов как косинус угла между двумя векторами. Это определяет,
направлены ли два вектора примерно в одном направлении. Таким образом, если угол между векторами равен 0 градусам, то косинусоидальное сходство равно 1.

Математическая формула: 
\begin{math}
    \label{cos_sim}
    \cos(A, B) = \frac{A * B}{\| A \| \times \| B \|,}
\end{math}
\cite{cosine}

Коэффициент корреляции Пирсона. Этот коэффициент показывает, насколько сильно и в каком направлении 
одна переменная (или компоненты одного вектора) зависит от другой.

Если говорить о сравнении векторов, то каждый вектор может быть интерпретирован как набор значений переменной. 
Корреляция Пирсона вычисляет степень линейной зависимости между соответствующими компонентами этих векторов.
\cite{lib:pearson}
\begin{equation}
    \label{pearson}
    r = \frac{\sum_{i=1}^{n} (x_i - \bar{x})(y_i - \bar{y})}{\sqrt{\sum_{i=1}^{n} (x_i - \bar{x})^2} \sqrt{\sum_{i=1}^{n} (y_i - \bar{y})^2}}
\end{equation}

Сходство Жаккара находит число общих элементов в них и делит найденное число на количество элементов обеих последовательностей.

Математическая формула: 
\begin{math}
    \label{jac_sim}
    Jac = \frac{len(A \cup B)}{len(A \cap B)}
\end{math}

\section{Получение результатов}

Используя полученные вектора, применяем методы их анализа для получения матрицы схожести до и после PCA,
которая, в свою очередь, используется для построения 3 графиков, отражающие результаты
исследования.

\section*{Вывод}

В разделе был рассмотрен процесс подготовки данных для метода главных компонент, 
включающий дополнение векторов словами для приведения их к единой размерности. 
Также были описаны три метрики: косинусоидальное сходство, коэффициент корреляции Пирсона и сходство Жаккара, 
с помощью которых проводится анализ близости между векторами документов. На основе этих метрик были получены матрицы схожести 
до и после применения PCA, результаты которых визуализированы на трёх графиках.

\clearpage
