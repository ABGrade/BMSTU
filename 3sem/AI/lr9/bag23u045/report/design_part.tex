\chapter{Конструкторская часть}

В рамках проектирования методов кластеризации основное внимание уделяется разработке алгоритмов, которые обеспечивают эффективность, точность и устойчивость к особенностям исходных данных. Это включает выбор оптимальных параметров, определение структуры данных и обеспечение интерпретируемости результатов.

\section{Структура реализации алгоритма \textit{k-means}}
1. Входные данные:
    \begin{itemize}
        \item $X = \{x_1, x_2, \ldots, x_n\}$ — множество объектов.
        \item $k$ — число кластеров.
    \end{itemize}

2. Инициализация:
    \begin{itemize}
        \item задание начальных центров кластеров $\mu_1, \mu_2, \ldots, \mu_k$ случайным образом или с использованием стратегии \textit{k-means++}.
    \end{itemize}

3. Цикл оптимизации:
    \begin{enumerate}
        \item назначение каждого объекта $x_j$ кластеру $C_i$ на основе ближайшего центра:
        \begin{math}
        C_i = \{x_j : \|x_j - \mu_i\| \leq \|x_j - \mu_l\|, \forall l \neq i\}.
        \end{math}\cite{lib:kmeans}
        \item пересчёт центров кластеров:
        \begin{math}
        \mu_i = \frac{1}{|C_i|} \sum_{x_j \in C_i} x_j.
        \end{math}\cite{lib:kmeans}
    \end{enumerate}

4. Критерий остановки:
    \begin{itemize}
        \item центры кластеров перестают изменяться или достигается максимальное число итераций.
    \end{itemize}

\section{Структура реализации алгоритма \textit{c-means}}
1. Входные данные:
    \begin{itemize}
        \item $X = \{x_1, x_2, \ldots, x_n\}$ — множество объектов.
        \item $k$ — число кластеров.
        \item $m > 1$ — параметр размытости.
    \end{itemize}

2. Инициализация:
    \begin{itemize}
        \item задание начальной матрицы принадлежностей $U = \{u_{ij}\}$ случайным образом с условием $\sum_{i=1}^k u_{ij} = 1$ для всех $j$.
    \end{itemize}

3. Цикл оптимизации:
    \begin{enumerate}
        \item пересчёт центров кластеров:
        \begin{math}
        \mu_i = \frac{\sum_{j=1}^n u_{ij}^m x_j}{\sum_{j=1}^n u_{ij}^m}.
        \end{math}\cite{lib:cmeans}
        \item обновление матрицы принадлежностей:
        \begin{math}
        u_{ij} = \frac{1}{\sum_{l=1}^k \left(\frac{\|x_j - \mu_i\|}{\|x_j - \mu_l\|}\right)^{\frac{2}{m-1}}}.
        \end{math}\cite{lib:cmeans}
    \end{enumerate}

4. Критерий остановки:
    \begin{itemize}
        \item изменения в матрице $U$ становятся меньше заданного порога или достигается максимальное число итераций.
    \end{itemize}

\section{Структура реализации иерархической кластеризации}
1. Входные данные:
    \begin{itemize}
        \item $X = \{x_1, x_2, \ldots, x_n\}$ — множество объектов.
        \item метрика расстояния (например, евклидово расстояние).
    \end{itemize}

2. Инициализация:
    \begin{itemize}
        \item каждому объекту $x_j$ соответствует отдельный кластер.
    \end{itemize}

3. Цикл объединения (агломеративный подход):
    \begin{enumerate}
        \item вычисление расстояния между всеми парами кластеров на основе выбранной метрики.
        \item объединение двух ближайших кластеров.
        \item обновление расстояний между новыми и оставшимися кластерами.
    \end{enumerate}

4. Результат:
    \begin{itemize}
        \item дендрограмма, представляющая иерархическую структуру кластеров.
    \end{itemize}

\clearpage
