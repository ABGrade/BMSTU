\chapter{Аналитическая часть}

Близость файлов - это близость двух векторов, представляющих файл. Другими словами наша задача состоит в определении насколько различны эти вектора, 
используя те или иные способы сравнения. Обычно сравнение происходит по шкале от 0 (абсолютно не похожи) до 1 (идентичны). 

Различных метрик сравнения файлов достаточно много, из них были использованы метод Жаккарда и косинусная мера близости. 

Основная сЛожность задачи заключается в построении этих самых векторов, которые должны максимально точно репрезентировать файл. 
В данной задаче используется, пожалуй, самый примитивный метод сравнения: по словам и их количеству. Этот способ я назвал примитивным, 
потому что он не отражает смысловой нагрузки, которая содержится в тексте. Некоторые слова могут содержать разные смыслы, в зависимости от контекста, 
в котором они употребляются. Например, слово "стекло" может иметь значения:
\begin{enumerate}
    \item Вещества и материала
    \item Глагола совершенного вида (образовано от слова стечь)
\end{enumerate}
К тому же роль играют и знаки препинания, которые могут придавать различные эмоциональные аспекты тексту. 
В данном случае, все знаки препинания считаются разделителями слов (в том числе и знак '-' который не всегда выступает в роли разделителя, например: что-то, где-то),
а потому не учитываются в построении вектора.

Существуют и различные способы учета слов в построении вектора. Мы можем учитывать слово полностью или использовать только его начальную форму, таким образом:
\begin{enumerate}
    \item Мама, мамы, маме, мам (все различные слова)
    \item Мама, мамы, маме, мам (н.ф. - мама, поэтому они все одинаковые)
\end{enumerate}

\section*{Вывод}

По каждому файлу строится два n-мерных вектора в формате: {слово : количество повторений}. 
Первый вектор учитывает все слово целиком, второй только начальную форму. Каждый знак препинания считается как разделитель и не учитывается при построении вектора.
Затем каждый вектор сравнивается с другими двумя методами: Жаккарда и косинусной мерой близости. 
По результатам сравнения строится график, который наглядно демонстрирует результаты сравнения.

\clearpage

