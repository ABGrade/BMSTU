\ssr{ЗАКЛЮЧЕНИЕ}

В ходе исследования удалось значительно сократить размерность вектора признаков — с 6917 до примерно 558–1454 признаков в зависимости от метода. 
Однако дальнейшее сокращение оказалось невозможным из-за проблем с визуализацией, так как центральные квадраты на тепловых картах перестают отображаться.

Все методы уменьшения размерности показали свою эффективность, за исключением метода, основанного на ковариации. 
При отборе признаков по ковариации даже уменьшение до около 6736 признаков привело к значительным потерям информации, 
что сделало этот метод непригодным для данных, с которыми проводилось исследование.

Таким образом, исследование показало, что предложенные методы позволяют уменьшить размерность вектора, 
но при этом сокращение недостаточно значительное, а возможные потери данных могут быть ощутимыми. 
Для достижения лучшего баланса между точностью и размерностью вектора стоит рассмотреть другие методы уменьшения размерности.

Из анализа сокращения размерности вектора признаков для различных тематических групп делается вывод, 
что эффективность методов отбора признаков варьируется в зависимости от тематики данных. 
Наибольшее сокращение размерности в большинстве тематик достигается при использовании метода отбора по Спирмену, 
что свидетельствует о его способности эффективно отбирать значимые признаки для конкретных типов данных. 
В то же время метод дисперсии, несмотря на свою эффективность в ряде случаев, часто оставляет значительно больше признаков, 
что указывает на его ограниченную применимость для некоторых тематических групп.

Метод ковариации, как и было отмечено ранее, показал наименьшую эффективность во всех случаях, 
так как оставил практически максимальное количество признаков, 
что приводит к значительным потерям информации при попытке более агрессивного сокращения. 
Это подтверждает, что для данных, используемых в исследовании, метод ковариации является наименее подходящим для уменьшения размерности.