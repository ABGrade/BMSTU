\chapter{Аналитическая часть}

Генерация текста представляет собой задачу создания последовательностей слов, которые соответствуют заданным характеристикам языка и контекста. 
Этот процесс базируется на методах статистического моделирования, алгоритмах машинного обучения и лингвистических принципах. 

\section{Цепи Маркова}
Цепи Маркова представляют собой стохастические процессы, в которых будущее состояние системы зависит только от текущего состояния, но не от предыдущих состояний. 
Основное предположение заключается в выполнении свойства Маркова:

\begin{math}
    P(X_{n+1} | X_1, X_2, \ldots, X_n) = P(X_{n+1} | X_n)
\end{math} \cite{lib:markov}.

В контексте генерации текста это означает, что вероятность появления следующего слова в последовательности зависит исключительно от одного или нескольких предыдущих слов. 
Модели на основе цепей Маркова позволяют эффективно генерировать текст, хотя их способность учитывать долгосрочные зависимости ограничена.

\section{n-граммные модели}
n-граммные модели используют фиксированные последовательности из n слов для предсказания следующего элемента. 
Для n-граммы вероятность последовательности рассчитывается как произведение условных вероятностей:

\begin{math}
    P(w_1, w_2, \ldots, w_n) = \prod_{i=1}^n P(w_i | w_{i-1}, w_{i-2}, \ldots, w_{i-n+1})
\end{math} \cite{lib:markov}.

Для построения таких моделей требуется создание корпуса текстов и вычисление частот встречаемости n-грамм. 
Увеличение значения n позволяет учитывать больше контекста, но увеличивает вычислительную сложность и требует больших объемов данных.

\section{SVO}
SVO (субъект-глагол-объект) — это структура предложений, характеризующаяся расположением элементов в указанной последовательности. 
В генерации текста использование SVO позволяет моделировать синтаксическую правильность предложений. 

Модели на основе SVO часто используют парсинг предложений для выделения этих компонентов, 
что помогает создавать более связный текст, соответствующий грамматическим правилам.

\section{Генерация текста на естественном языке}
Современные технологии генерации текста на естественном языке (ЕЯ) используют глубокое обучение и большие языковые модели. 
Эти подходы основываются на трансформерах, таких как GPT, BERT и их аналоги. 
Основной принцип работы заключается в обучении модели на огромных объемах текстовых данных для предсказания следующего слова 
или генерации связного текста в рамках заданного контекста. 

Процесс генерации текста начинается с ввода начального текста или запроса, который служит контекстом. 
Затем модель предсказывает вероятностное распределение для каждого следующего слова, выбирая наиболее подходящее из них. 

\section{Модель Qwen-2.5}
Qwen-2.5 является одной из современных языковых моделей, разработанных для генерации текста высокого качества. 
Она сочетает в себе возможности обработки длинного контекста, управления стилем текста и учета семантических связей. 
Модель использует архитектуру трансформера с оптимизацией для большего числа параметров, что позволяет ей генерировать текст 
с высокой степенью связности и стилистической корректности.

Ключевые особенности Qwen-2.5 включают:
\begin{itemize}
    \item многозадачное обучение, что позволяет ей эффективно справляться с различными задачами, включая генерацию текста, перевод и резюмирование.
    \item улучшенные механизмы внимания, обеспечивающие точный учет контекста.
    \item поддержку заданных стилевых ограничений, таких как формальный или разговорный стиль.
\end{itemize}

Эта модель широко используется в практических приложениях, где требуется сочетание точности и естественности текста.

\section*{Вывод}

Эти подходы демонстрируют эволюцию технологий, направленных на создание текстов, которые максимально приближены к человеческому уровню. 
Комбинирование различных методов позволяет достичь высокой точности, связности и стилистической выразительности текста.

\clearpage