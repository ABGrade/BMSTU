\chapter{Конструкторская часть}

\section{Подготовка данных}
Название программы: preproc

Читает текст из документов с различными форматами в папке "тексты". По данному тексте строит два словаря (вектора) 
в формате {слово : кол-во повторений}, где в первом словаре сохраняется слово целиком, а во втором его начальная форма, 
оба вектора сохраняются в подпапку с именами vectors и norm vectors соответственно для дальнейшего анализа.

\section{Метрики}

Название программы: metrics.py

Содержит используемые для анализа метрики: метод Жаккарда и косинусной мерой близости.
Сходство Жаккара — это простая, но иногда мощная метрика сходства. Даны две последовательности A и B: 
находим число общих элементов в них и делим найденное число на количество элементов обеих последовательностей.

Математическая формула: 
\begin{math}
    \label{jac_sim}
    Jac = \frac{len(A \cup B)}{len(A \cap B)}
\end{math}

Косинусоидальное сходство вычисляет сходство двух векторов как косинус угла между двумя векторами. Это определяет,
направлены ли два вектора примерно в одном направлении. Таким образом, если угол между векторами равен 0 градусам, то косинусоидальное сходство равно 1.

Математическая формула: 
\begin{math}
    \label{cos_sim}
    \cos(A, B) = \frac{A * B}{\| A \| \times \| B \|,}
\end{math}

\section{Получение результатов}

Название программы: main.py

Использует полученные векторы и методы их анализа для получения матрицы схожести, которая, в свою очередь, 
используется для построения 4 графиков, отражающие результаты исследования. Графики сохраняются в папку "results" рядом с папкой "тексты".

\section*{Вывод}

Подобная структура файлов, позволяет удобно ориентироваться между функциями каждой отдельной программы. Каждый файл выполняется свою важную функцию в общей программе.

\clearpage
