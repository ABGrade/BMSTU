\chapter{Технологическая часть}

\section{Средства написания программы}
Для реализации программного обеспечения были использованы следующие средства:

\begin{itemize}
    \item cреда разработки PyCharm 2023 community edition \cite{lib:pycharm}
    \item язык разработки Python 3.10 \cite{lib:python}
\end{itemize}
	
Используемые библиотеки:
\begin{itemize}
    \item os — для доступа к файлам системы и создания папок/файлов \cite{lib:os}
    \item numpy — для быстрой работы с массивами, сохранения промежуточных данных и связи их с другими библиотеками \cite{lib:numpy}
    \item matplotlib.pyplot — для отображения графиков \cite{lib:matplotlib}
    \item seaborn — для построения графиков \cite{lib:seaborn}
    \item sklearn.metrics.pairwise.cosine\_similarity — косинусная мера близости \cite{lib:sklearn}
    \item sklearn.metrics.jaccard\_score — метод Жаккарда \cite{lib:sklearn}
    \item pandas — все использованные статистические функции \cite{lib:pandas}
    \item pathlib.Path — доступ к файлам системы \cite{lib:pathlib}
\end{itemize}

\section{Загрузка векторов документов}

\begin{lstlisting}[label=load_vecs, caption={Загрузка векторов документов}]
    def load_vectors_from_directory(base_dir):
        vectors = []
        files = []
        for file_path in base_dir.rglob('vectors/*.txt'):
            files.append(file_path.name)
            vector = np.loadtxt(file_path)
            vectors.append(vector)
        return np.vstack(vectors), files
\end{lstlisting}

\section{Выбор признаков}

\begin{lstlisting}[label=del_mean, caption={Выбор признаков по математическому ожиданию}]
    def to_del_by_mean(df: pd.DataFrame):
        threshold = 0.04
        df_mean = df.mean()
        list_to_remove = df_mean[df_mean <= threshold].index.to_list()
        return list_to_remove
\end{lstlisting}

\begin{lstlisting}[label=del_var, caption={Выбор признаков по дисперсии}]
    def to_del_by_dispersion(df: pd.DataFrame):
        threshold = 0.99
        df_var = df.var()
        mean_df_var = df_var.mean()
        lower_bound = mean_df_var * (1 - threshold)
        upper_bound = mean_df_var * (1 + threshold)
        list_to_remove = df_var[(df_var > lower_bound) & (df_var < upper_bound)].index.to_list()
        return list_to_remove
\end{lstlisting}

\begin{lstlisting}[label=del_cov, caption={Выбор признаков по ковариации}]
    def to_del_by_covariance(df: pd.DataFrame):
        threshold = 0.6
        df_cov = df.cov().abs()
        df_cov = df_cov.where(np.tril(df_cov, k=-1) > threshold)
        to_drop = set()
        for col in df_cov.columns:
            if col not in to_drop:
                correlated_features = df_cov[col].dropna().index
                to_drop.update(correlated_features.difference([col]))
        return to_drop
\end{lstlisting}

\begin{lstlisting}[label=del_pearson, caption={Выбор признаков по Пирсону}]
    threshold = 0.99
    df_pearson = df.corr(method='pearson').abs()
    df_pearson = df_pearson.where(np.tril(df_pearson, k=-1) > threshold)
    to_drop = set()
    for col in df_pearson.columns:
        if col not in to_drop:
            correlated_features = df_pearson[col].dropna().index
            to_drop.update(correlated_features.difference([col]))
    return to_drop
\end{lstlisting}

\begin{lstlisting}[label=del_spearman, caption={Выбор признаков по Спирмену}]
    threshold = 0.99
    df_pearson = df.corr(method='pearson').abs()
    df_pearson = df_pearson.where(np.tril(df_pearson, k=-1) > threshold)
    to_drop = set()
    for col in df_pearson.columns:
        if col not in to_drop:
            correlated_features = df_pearson[col].dropna().index
            to_drop.update(correlated_features.difference([col]))
    return to_drop
\end{lstlisting}

\section{Получение результатов}

\begin{lstlisting}[label=build_heatmaps, caption={Построение тепловых карт}]
    def build_heatmap(data, savepath, xlables, ylabels):
        plt.figure(figsize=(19.2, 10.8))
        sea.heatmap(data, xticklabels=xlables, yticklabels=ylabels, square=True, linewidths=.3, cmap="YlGnBu")
        plt.savefig(savepath, dpi=100)
        plt.clf()
\end{lstlisting}

\begin{lstlisting}[label=get_res, caption={Внесение измений и построение тепловых карт}]
    to_drop_by_dispersion = determining.to_del_by_dispersion(df)
    vectors_updated_dispersion = np.delete(vectors, list(to_drop_by_dispersion), axis=1)
    pearson_matrix, cosine_matrix, jac_matrix = get_corr_matrix(vectors_updated_dispersion)
    build_heatmap(pearson_matrix, RES_DIR / "Heatmap_dispersion_pearson.png", files, files)
    build_heatmap(cosine_matrix, RES_DIR / "Heatmap_dispersion_cosine.png", files, files)
    build_heatmap(jac_matrix, RES_DIR / "Heatmap_dispersion_jac.png", files, files)

    to_drop_by_mean = determining.to_del_by_mean(df)
    vectors_updated_mean = np.delete(vectors, list(to_drop_by_mean), axis=1)
    pearson_matrix, cosine_matrix, jac_matrix = get_corr_matrix(vectors_updated_mean)
    build_heatmap(pearson_matrix, RES_DIR / "Heatmap_mean_pearson.png", files, files)
    build_heatmap(cosine_matrix, RES_DIR / "Heatmap_mean_cosine.png", files, files)
    build_heatmap(jac_matrix, RES_DIR / "Heatmap_mean_jac.png", files, files)

    to_drop_by_covariance = determining.to_del_by_covariance(df)
    vectors_updated_covariance = np.delete(vectors, list(to_drop_by_covariance), axis=1)
    pearson_matrix, cosine_matrix, jac_matrix = get_corr_matrix(vectors_updated_covariance)
    build_heatmap(pearson_matrix, RES_DIR / "Heatmap_covariance_pearson.png", files, files)
    build_heatmap(cosine_matrix, RES_DIR / "Heatmap_covariance_cosine.png", files, files)
    build_heatmap(jac_matrix, RES_DIR / "Heatmap_covariance_jac.png", files, files)

    to_drop_by_pearson = determining.to_del_by_pearson(df)
    vectors_updated_pearson = np.delete(vectors, list(to_drop_by_pearson), axis=1)
    pearson_matrix, cosine_matrix, jac_matrix = get_corr_matrix(vectors_updated_pearson)
    build_heatmap(pearson_matrix, RES_DIR / "Heatmap_pearson_pearson.png", files, files)
    build_heatmap(cosine_matrix, RES_DIR / "Heatmap_pearson_cosine.png", files, files)
    build_heatmap(jac_matrix, RES_DIR / "Heatmap_pearson_jac.png", files, files)

    to_drop_by_spearman = determining.to_del_by_spearman(df)
    vectors_updated_spearman = np.delete(vectors, list(to_drop_by_spearman), axis=1)
    pearson_matrix, cosine_matrix, jac_matrix = get_corr_matrix(vectors_updated_spearman)
    build_heatmap(pearson_matrix, RES_DIR / "Heatmap_spearman_pearson.png", files, files)
    build_heatmap(cosine_matrix, RES_DIR / "Heatmap_spearman_cosine.png", files, files)
    build_heatmap(jac_matrix, RES_DIR / "Heatmap_spearman_jac.png", files, files)
\end{lstlisting}

\section*{Вывод}

В технологической части были разработаны и реализованы программные методы для автоматизированного отбора признаков и анализа близости документов.
Методы для отбора признаков включают вычисления на основе математического ожидания, дисперсии, ковариации, а также корреляций по Пирсону и Спирмену. 
Эти методы позволили выявить малозначимые и дублирующие признаки, которые затем исключались из векторов документов для повышения точности анализа.

На завершающем этапе была построена матрица корреляции и визуализированы тепловые карты взаимосвязей между документами,
что позволило наглядно представить результаты отбора признаков и оценить качество полученных данных.

\clearpage
