\chapter{Конструкторская часть}

Программа состоит из двух основных частей: обнаружение ромба и обнаружение лошади. 
Для каждого объекта используется индивидуальный подход, основанный на особенностях их визуальных признаков.

\section{Обнаружение ромба}

Алгоритм поиска ромбов в изображении основывается на анализе геометрических контуров. 
На первом этапе изображение преобразуется в оттенки серого для упрощения обработки. 
Затем применяется адаптивная бинаризация с использованием метода Гаусса, что позволяет выделить основные контуры объектов. 
Полученные контуры анализируются на предмет их формы. Для этого используется аппроксимация полигонов, определяющая количество углов у замкнутого контура.

Если контур имеет четыре угла и является выпуклым, выполняется дальнейшая проверка. 
Рассчитываются длины сторон и углы между ними, чтобы определить, соответствует ли фигура характеристикам ромба. 
Дополнительно проверяются диагонали фигуры на их соотношение длины, что позволяет отфильтровать ложные срабатывания. 

\section{Обнаружение лошади}

Для обнаружения лошади используется модель глубокого обучения YOLO (You Only Look Once). 
Программа загружает предварительно обученные веса и конфигурацию модели, после чего подготавливает входное изображение для обработки. 

Модель предсказывает координаты объектов на изображении и их классы. 
Из всех предсказанных объектов отбираются только те, которые соответствуют классу \textit{horse}. 
Для каждой найденной области вычисляется вероятность принадлежности к данному классу. 
На этапе постобработки применяется алгоритм подавления перекрывающихся прямоугольников (NMS), 
который удаляет избыточные рамки и оставляет только наиболее вероятные детекции. 

\section*{Вывод}

Разработанный алгоритм демонстрирует комплексный подход к решению задачи детектирования объектов на изображении. 
Использование классических методов обработки изображения для поиска ромбов и современных моделей глубокого обучения 
для обнаружения лошадей обеспечивает высокую точность и надёжность работы программы. 

\clearpage
