\ssr{ЗАКЛЮЧЕНИЕ}

Разработанный алгоритм продемонстрировал возможность решения задачи автоматического поиска заданных объектов на изображениях с использованием библиотеки OpenCV. 
Несмотря на общее подтверждение состоятельности предложенного метода, 
проведённый анализ результатов выявил ряд ограничений и проблем, которые необходимо учитывать.

Результаты детектирования лошадей показывают, что использование моделей глубокого обучения, 
таких как YOLO, позволяет эффективно находить объекты данного класса. 
Однако ограничение входного изображения размером 416x416 пикселей, накладываемое архитектурой сети, 
приводит к потере некоторых деталей сцены, что может снижать качество обнаружения. 

С точки зрения детектирования ромбов, применение классических методов обработки изображения дало неоднозначные результаты. 
Хотя часть объектов была успешно обнаружена, алгоритм продемонстрировал чувствительность к помехам и вариациям формы объектов. 
В ряде случаев наблюдались ложные срабатывания, а также пропуски очевидных ромбов, 
что может быть связано с особенностями реализации метода аппроксимации контуров и выбора пороговых значений для проверки геометрических характеристик.

Таким образом, для повышения точности и надёжности алгоритма требуется провести дополнительные исследования. 
В частности, возможным направлением улучшения является использование более мощных архитектур глубокого обучения 
для обработки ромбов или гибридного подхода, комбинирующего классические методы с нейронными сетями. 
Исследовать масштабирование изображений на производительность алгоритма, 
а также оптимизировать пороговые параметры для уменьшения числа ложных срабатываний.

Подводя итог, предложенный метод обладает потенциалом для решения задач компьютерного зрения, 
однако требует дальнейшей доработки для достижения более высоких показателей точности и универсальности.
