\chapter{Исследовательская часть}

Цель исследования — оценить возможности цепей Маркова для генерации текстов.

\section{Оборудование}

Характеритстики ноутбука:
\begin{itemize}
	\item процессор intel-core i5-12500H \cite{lib:intel}
	\item ОЗУ 16 Гб DDR4
	\item ОС Windows 11 \cite{lib:windows}
\end{itemize}

\section{Результаты исследования}

\begin{longtable}{|p{4cm}|p{12cm}|}
    \caption{Сгенерированный текст на основе 2-граммы} \\

    \hline
    \textbf{Начальное слово} & \textbf{Сгенерированный текст} \\ \hline
    \endfirsthead

    \hline
    \textbf{Начальное слово} & \textbf{Сгенерированный текст} \\ \hline
    \endhead

    \hline
    \endfoot
    "и" & и др получения органических соединений основоположником которой был а м бутлерова в в китае производился фарфор в алхимический период до нач 13 в стала применяться а в 15 в и производиться селитра в период грудного вскармливания альбумин глобулины если кожа и белки глаз приобрели желтоватый оттенок это повод проверить работу \\ \hline
    "в" & в микроэлементах показания к проведению контроль состояния здоровья функционального состояния и здоровья в целом удельный вес различных элементов комплексного лечения зависит от врача и больного зависит та психологическая совместимость которая во многом будет зависеть от доступности сложных органических молекул и стоимости их производства \\ \hline
    "с" & с направлением в биохимическую лабораторию измерение различных показателей крови занимает всего несколько часов но почти всегда заключение отдается пациенту следующим утром оценкой результатов должен заниматься врач знающий о соответствии характерных нарушений конкретным заболеваниям самостоятельное ознакомление пациента с результатом анализов позволит заметить нарушение но диагностировать заболевание по всем имеющимся данным способен \\ \hline
    "как" & как и перед сдачей анализов по концентрации липопротеидов либо холестерина следует выдержать 12 14 часовое голодание определение мочевой кислоты нарушения углеводного обмена концентрацию ферментов исключают или фиксируют изменения гомеостаза у пациентов с эпилепсией приступая к диагностике врач прежде всего физические методы исследования спектроскопия в первую очередь следует обратиться в службу \\ \hline
    "химии" & химии тесно связана с общей историей химии а вместе с ней с историей естествознания и историей человеческой цивилизации составные разделы истории неорганической химии стали использоваться такие понятия как введенная л полингом электроотрииательность ионные и ковалентные радиусы степень окисления кислоты и нагревании до 1700с этилен представляет собой бесцветный почти нерастворимый в \\ \hline
    "веществ" & веществ пигментов энзимов изменение каждого из показателей свидетельствует о проблемах со здоровьем занимающихся физической культурой испортом имеют своей целью допуск к спортивным занятиям систематическое изучение влияния этих занятий на физическое развитие состояние здоровья физическое развитие и функциональные возможности спортсмена при этих обследованиях нужно выяснять также степень сдвигов в состоянии организма \\ \hline
\end{longtable}

\begin{longtable}{|p{4cm}|p{12cm}|}
    \caption{Сгенерированный текст на основе 3-граммы}
    \hline
    \textbf{Начальное слово} & \textbf{Сгенерированный текст} \\ \hline
    \endfirsthead
    \hline
    \textbf{Начальное слово} & \textbf{Сгенерированный текст} \\ \hline
    \endhead
    "и" &  и др получить информацию о метаболизме обмен липидов белков углеводов выяснить потребность в микроэлементах показания к проведению контроль состояния здоровья не реже 1 раза в год надо следить за тем чтобы в течение года общее количество взятой крови у человека в том числе для продвижения личного бренда \\ \hline
    "в" &  в микроэлементах показания к проведениюконтроль состояния здоровья не реже 1 раза в год для отдельных заболеваний биохимия является единственной возможностью для объективной диагностики кроме стандартного биохимического анализа проводится исследование специфических показателей использующихся в генетике эндокринологии педиатрии спортивной медицине значения показателей кровь в количестве 5 10 миллилитров берут из вены затем \\ \hline
    "с" &  с направлением в биохимическую лабораторию измерение различных показателей крови занимает всего несколько часов но почти всегда заключение отдается пациенту следующим утром оценкой результатов должен заниматься врач знающий о соответствии характерных нарушений конкретным заболеваниям самостоятельное ознакомление пациента с результатом анализов позволит заметить нарушение но диагностировать заболевание по всем имеющимся данным способен    \\ \hline
    "как" &  как и расспрос в целом — не просто перечень вопросов и ответов на них от стиля беседы врача и больного зависит та психологическая совместимость которая во многом определяет конечную цель — облегчение состояния пациента данные анамнеза сведения о развитии болезни условиях жизни перенесённых заболеваниях операциях травмах беременностях хронической патологии аллергических    \\ \hline
    "химии" &  химии тесно связанная помимо органической с другими разделами химии аналитической химией коллоидной химией биохимией кристаллохимией физической химией химической термодинамикой электрохимией радиохимией химической физикой и др на стыке неорганической и органической химии различаются это позволяет проще систематизировать методы и способы исследования в каждой из отраслей неорганическая химия изучает общие правила и    \\ \hline
    "веществ" & веществ пигментов энзимов изменение каждого из показателей свидетельствует о проблемах со здоровьем \\ \hline
\end{longtable}

\begin{longtable}{|p{4cm}|p{12cm}|}
    \caption{Сгенерированный текст на основе 5-граммы}
    \hline
    \textbf{Начальное слово} & \textbf{Сгенерированный текст} \\ \hline
    \endfirsthead
    \hline
    \textbf{Начальное слово} & \textbf{Сгенерированный текст} \\ \hline
    \endhead
    "и" &   и др получить информацию о метаболизме обмен липидов белков углеводов выяснить потребность в микроэлементах показания к проведению контроль состояния здоровья не реже 1 раза в год надо следить за тем чтобы в течение года общее количество взятой крови у человека в том числе и в диагностических целях не превышало скорость    \\ \hline
    "в" &  в микроэлементах показания к проведениюконтроль состояния здоровья не реже 1 раза в год надо следить за тем чтобы в течение года общее количество взятой крови у человека в том числе и в диагностических целях не превышало скорость образования эритроцитов перенесенные инфекционные или соматические заболевания перед проведением биохимического анализа крови человека    \\ \hline
    "с" &   с направлением в биохимическую лабораторию измерение различных показателей крови занимает всего несколько часов но почти всегда заключение отдается пациенту следующим утром оценкой результатов должен заниматься врач знающий о соответствии характерных нарушений конкретным заболеваниям самостоятельное ознакомление пациента с результатом анализов позволит заметить нарушение но диагностировать заболевание по всем имеющимся данным способен    \\ \hline
    "как" &   как и расспрос в целом — не просто перечень вопросов и ответов на них от стиля беседы врача и больного зависит та психологическая совместимость которая во многом определяет конечную цель — облегчение состояния пациента данные анамнеза сведения о развитии болезни условиях жизни перенесённых заболеваниях операциях травмах беременностях хронической патологии аллергических    \\ \hline
    "химии" &   химии тесно связанная помимо органической с другими разделами химии аналитической химией коллоидной химией биохимией кристаллохимией физической химией химической термодинамикой электрохимией радиохимией химической физикой и др на стыке неорганической и органической химии находится химия металлоорганических соединений и элементоорганических соединений неорганическая химия соприкасается с геолого минералогическими науками прежде всего с геохимией и    \\ \hline
    "веществ" &  веществ пигментов энзимов изменение каждого из показателей свидетельствует о проблемах со здоровьем    \\ \hline
\end{longtable}

\begin{longtable}{|p{4cm}|p{12cm}|}
    \caption{Сгенерированный текст на основе 2-граммы SVO}
    \hline
    \textbf{Начальное слово} & \textbf{Сгенерированный текст} \\ \hline
    \endfirsthead
    \hline
    \textbf{Начальное слово} & \textbf{Сгенерированный текст} \\ \hline
    \endhead
    "кошка" &   кошка съела мышку    \\ \hline
    "мышку" &   мышку съела кошка    \\ \hline
\end{longtable}

\begin{longtable}{|p{4cm}|p{12cm}|}
    \caption{Сгенерированный текст на основе 2-граммы не SVO}
    \hline
    \textbf{Начальное слово} & \textbf{Сгенерированный текст} \\ \hline
    \endfirsthead
    \hline
    \textbf{Начальное слово} & \textbf{Сгенерированный текст} \\ \hline
    \endhead
    "кошка" &   кошка съела мышку кошка съела мышку кошка съела мышку кошка съела мышку кошка съела мышку кошка съела мышку кошка съела мышку кошка съела мышку кошка съела мышку кошка съела мышку кошка съела мышку кошка съела мышку кошка съела мышку кошка съела мышку кошка съела мышку кошка съела мышку кошка съела    \\ \hline
    "мышку" &   мышку кошка съела мышку кошка съела мышку кошка съела мышку кошка съела мышку кошка съела мышку кошка съела мышку кошка съела мышку кошка съела мышку кошка съела мышку кошка съела мышку кошка съела мышку кошка съела мышку кошка съела мышку кошка съела мышку кошка съела мышку кошка съела мышку кошка    \\ \hline
    "съела" &  съела кошка мышку съела кошка мышку съела кошка мышку съела кошка мышку съела кошка мышку съела кошка мышку съела кошка мышку съела кошка мышку съела кошка мышку съела кошка мышку съела кошка мышку съела кошка мышку съела кошка мышку съела кошка мышку съела кошка мышку съела кошка мышку съела кошка    \\ \hline
\end{longtable}

\section*{Вывод}

В ходе исследования был выполнен анализ моделей генерации текстов на основе n-грамм (2-граммы, 3-граммы, 5-граммы) 
и модификации 2-грамм с учетом SVO-структуры (подлежащее–сказуемое–дополнение). Максимальная длина предложения была выбрана 50.
Начальные слова (не считая части SVO) были выбраны по мере самых встречающихся слов в текстах. 
Слова "и", "в", "с" оказались в топ 3, а "как", "химии" и "веществ" на усмотрение студента.

В текстах на основе 2-грамм заметно отсутствие глобального контекста, что приводит к бессмысленным последовательностям слов, несмотря на локальную связность. 
Это делает такие тексты слабо воспринимаемыми для человека, так как они не соответствуют привычным языковым паттернам.
Тексты, построенные на основе 3-грамм, демонстрируют лучшую связность и некоторую осмысленность благодаря большему объему контекстной информации. 
Однако в них также часто встречаются повторы и смысловые пробелы, что ограничивает их "человечность". 
Модели 5-грамм дают наиболее осмысленные результаты за счет учета более широкого контекста, что приближает тексты к привычным для человека структурам. 
Тем не менее, отсутствие глубокой семантической обработки и знаний о мире ограничивает их практическую применимость.

В части SVO использование строгого порядка слов подлежащее–сказуемое–дополнение позволяет создавать короткие осмысленные фразы, которые легко воспринимаются человеком. 
При этом нарушение строгого порядка приводит к опасности избыточной повторяемости и потери связности, что может затруднить восприятие текста. 
Например, генерация без контроля структуры SVO показала склонность модели к многократным бессмысленным повторам ("кошка съела мышку..."). 

По мере роста параметров (от 0.5B до 7B), наблюдается улучшение в понимании языковых конструкций, включая синтаксис, грамматику и семантику. 
Модель демонстрируют более глубокое понимание контекста, повышается их способность к логическому рассуждению и разрешению неоднозначностей.

В контексте генерации текста, увеличивается связность, когерентность и стилистическая гибкость. 
Креативность, оригинальность и разнообразие текстов также возрастают, наряду со способностью удерживать и использовать длинный контекст.

Модели с большим количеством параметров демонстрируют улучшенные результаты в решении задач обработки естественного языка, 
включая ответы на вопросы, перевод, суммирование и генерацию кода. 
Также наблюдается повышение эффективности в переключении между различными типами задач.

В области диалоговых систем, модели большего размера обеспечивают более естественное взаимодействие с пользователем. 
Улучшается способность удерживать контекст беседы и адаптироваться к собеседнику, а также к обработке неоднозначных реплик.

Увеличение масштаба моделей сопряжено с возрастанием требований к вычислительным ресурсам, энергопотреблению и снижением прозрачности их работы. 
Также возрастает потенциал для создания дезинформации.

При переходе от модели 0.5B к 7B, наблюдается существенное улучшение в лингвистической компетенции, 
генерации связного текста и способности решать сложные задачи. 
Модель 0.5B демонстрирует лишь базовые навыки, в то время как модель 7B способна к более сложному анализу и креативному синтезу информации.

\clearpage
