\chapter{Конструкторская часть}

Данная программа разработана для исследования возможностей генерации текстов с помощью цепей Маркова. 
Основной целью является анализ влияния структуры исходных данных и начальных условий на осмысленность и качество генерируемых текстов. 
Программа решает следующие задачи:
\begin{enumerate}
    \item генерация текстов на основе цепей Маркова с использованием n-грамм разной длины (\(n = 2, 3, 5\)).
    \item исследование особенностей генерации текста при ограниченном наборе обучающих данных, 
    таких как предложения «кошка съела мышку» и «мышку съела кошка», с вариациями порядка слов.
\end{enumerate}

\section{Принцип работы}

\subsection{Генерация текстов по цепям Маркова}
Тексты считываются из заданной директории, после чего они подвергаются токенизации. 
Для каждого текста извлекаются n-граммы, представляющие собой последовательности из \(n\) слов. 
Эти n-граммы используются для построения цепи Маркова.

Цепь Маркова строится на основе сформированных n-грамм. 
Каждое состояние цепи представляет собой последовательность из \(n\) слов,
 а возможные переходы к следующему состоянию определяются следующим словом из текста.

На основе созданной цепи Маркова программа генерирует новые тексты. 
Генерация начинается с заданного начального слова, после чего цепь формирует последовательности, выбирая вероятные переходы. 
Для оценки создаются тексты с различными начальными условиями.

\subsection{Исследование с фиксированными предложениями}
Программа анализирует случаи, когда обучающая выборка ограничена двумя предложениями: 
«кошка съела мышку» и «мышку съела кошка». 
Для этих данных проводится генерация текстов как при сохранении строгого порядка слов (SVO), так и при использовании всех возможных перестановок слов.

\section*{Вывод}
Программа демонстрирует потенциал цепей Маркова для генерации текстов на основе заданного обучающего корпуса. 

\clearpage
