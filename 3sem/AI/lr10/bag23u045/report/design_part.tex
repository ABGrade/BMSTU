\chapter{Конструкторская часть}

Для анализа близости используется вектор документа, способ получения которого описывался в лабораторной работе №7. 
Метод получения векторов документа ничем не отличается от описанного в нём.
Также из той же лабораторной работы используются метрики для получения матрицы близости документов.
Поэтому в данной лабораторной работе они подробно описаны не будут.

\section{Гипотезы}

\subsection{Математическое ожидание}
По результатам расчета математического ожидания предполагается, что признаки со средним значением близким к нулю (в данном случае близким к нуля считается тот, что встречается в 2 документах из 50),
не несут важной информации, поэтому могут быть удалены.

\subsection{Дисперсия}
Дисперсия позволяет выявить признаки с низкой изменчивостью. 
Признаки с низкой изменчивостью считаться те, что отличаются от среднего по всем признакам не более чем на одно значение среднего. 
Предполагается, что они не несут значимой информации для различения документов, так как мало отличаются друг от друга.

\subsection{Ковариация}
Так как важен лишь сам факт зависимости одного признака от другого, а не направление зависимости, было принять решения брать ковариацию по модулю.
Предполагается, что высокая ковариация может говорить о том, что признаки дублируют друг друга. Поэтому один из них можно исключить, так как они отражают схожую информацию о документе.

\subsection{Корреляция Пирсона}
Как и при ковариации, важен лишь сам факт зависимости, поэтому корреляция будет браться по модулю. 
Предполагется, что признаки с высокой корреляцией Пирсона можно объединить или исключить один из них, так как они отражают схожую линейную информацию.

\subsection{Корреляция Спирмена}
Ситуация аналогична корреляции Пирсона.

\section{Получение результатов}
Используя выше описанные методы, получим список признаков, которые могут быть удалены.
После обновления векторов можно будет приступить к построению матрицы корреляции и получению хитмапов.

\section*{Вывод}
Гипотезы, выдвинутые в исследовании, касаются отбора признаков для улучшения качества представления документов. 
В результате, путем исключения нерелевантных признаков обновленные векторы документов позволят построить матрицу корреляции и 
визуализировать взаимосвязи с помощью тепловой карты, что улучшит точность анализа текстовых данных.

\clearpage
