\ssr{ЗАКЛЮЧЕНИЕ}

Цель исследования достигнута: выполнено уменьшение вектора файла методом главных компонент, 
что позволило сохранить ключевую информативность данных и провести сравнительный анализ векторов до и после преобразования.

Построенные графики показывают степень изменений в векторах. 
Несмотря на различия в структуре нормализированных и обычных векторов, значительная часть исходных взаимосвязей сохраняется. 
Применение метода PCA выявило более выраженные связи между документами, что наглядно отражено на тепловых картах.

Самыми информативными метриками для анализа оказались косинусная мера близости и корреляция Пирсона. 
Эти метрики показывают практически идентичные результаты, как на тепловых картах, так и при сравнении с матрицами до уменьшения размерности. 
Метод Жаккарда показал менее согласованные результаты, что затрудняет их интерпретацию по сравнению с предыдущими метриками.