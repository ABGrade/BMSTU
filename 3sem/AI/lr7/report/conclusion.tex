\ssr{ЗАКЛЮЧЕНИЕ}

Удалось достигнуть выдвинутой цели: исследовать файлы на похожесть.

Была выполнена поставленная задача: успешно построены вектора, репрезентующие содержимое каждого файла в двух варинатах:
\begin{itemize}
    \item с выделением словоформ
    \item с выделением начальных форм слов
\end{itemize}

В результате получили графики, на которых можно проследить близость файлов. Эти результаты подчеркивают важность выбора подходящей метрики для выбранной задачи. 
Так метод Жаккара практически не отражает схожесть документов, что идет в разрез с косинусной мерой.

Помимо прочего четкто прослеживается как нормализированный график отличается от обычного. Рисунок не изменяется, а вот оттенки цветов изменяются, преимущественно в большую сторону.
То есть нормализированные вектора лишь подчеркивают уже вырисовывающуюся близость.