\chapter{Исследовательская часть}

Цель исследования — обнаружить на фотографии лошадей и ромбы.

\section{Оборудование}

Характеритстики ноутбука:
\begin{itemize}
	\item процессор intel-core i5-12500H \cite{lib:intel}
	\item ОЗУ 16 Гб DDR4
	\item ОС Windows 11 \cite{lib:windows}
\end{itemize}

\section{Обнаруженные ромбы}

\begin{figure}[H]
    \centering
    \includegraphics[width=1\linewidth]{C:/MGTU/baseAI/lr4/bag/report/images/diamonds/1.jpg}
    \caption{Изображение 1}
\end{figure}

\begin{figure}[H]
    \centering
    \includegraphics[width=1\linewidth]{C:/MGTU/baseAI/lr4/bag/report/images/diamonds/5.jpg}
    \caption{Изображение 5}
\end{figure}

\begin{figure}[H]
    \centering
    \includegraphics[width=1\linewidth]{C:/MGTU/baseAI/lr4/bag/report/images/diamonds/10.jpg}
    \caption{Изображение 10}
\end{figure}

\begin{figure}[H]
    \centering
    \includegraphics[width=1\linewidth]{C:/MGTU/baseAI/lr4/bag/report/images/diamonds/11.jpg}
    \caption{Изображение 11}
\end{figure}

\begin{figure}[H]
    \centering
    \includegraphics[width=1\linewidth]{C:/MGTU/baseAI/lr4/bag/report/images/diamonds/12.jpg}
    \caption{Изображение 12}
\end{figure}

\begin{figure}[H]
    \centering
    \includegraphics[width=1\linewidth]{C:/MGTU/baseAI/lr4/bag/report/images/diamonds/13.jpg}
    \caption{Изображение 13}
\end{figure}

\begin{figure}[H]
    \centering
    \includegraphics[width=1\linewidth]{C:/MGTU/baseAI/lr4/bag/report/images/diamonds/15.jpg}
    \caption{Изображение 15}
\end{figure}

\begin{figure}[H]
    \centering
    \includegraphics[width=1\linewidth]{C:/MGTU/baseAI/lr4/bag/report/images/diamonds/16.jpg}
    \caption{Изображение 16}
\end{figure}

\begin{figure}[H]
    \centering
    \includegraphics[width=1\linewidth]{C:/MGTU/baseAI/lr4/bag/report/images/diamonds/17.jpg}
    \caption{Изображение 17}
\end{figure}

\begin{figure}[H]
    \centering
    \includegraphics[width=1\linewidth]{C:/MGTU/baseAI/lr4/bag/report/images/diamonds/19.jpg}
    \caption{Изображение 19}
\end{figure}

\section{Обнаруженные лошади}

\begin{figure}[H]
    \centering
    \includegraphics[width=1\linewidth]{C:/MGTU/baseAI/lr4/bag/report/images/horses/1.jpg}
    \caption{Изображение 1}
\end{figure}

\section{Вывод}

Полученные изображения в общем случае показывают состоятельность данных метод, 
хотя и с далеко не идеальными результатами, часть из которых можно объяснить тем, 
что, например, для поиска лошадей используется лишь часть изображения подходящего
для cv2 формата (в данном случае 416х416), что не дает полной картины.

Что касается ромбов, то результаты далеки от идеальных, многие объекты были ошибочно обнаружены, 
а в некоторых случаях очевидные ромбы (с точки зрения человека) не были обнаружены, 
хотя правильные результататы, в некоторых случаях, были получены.

\clearpage
