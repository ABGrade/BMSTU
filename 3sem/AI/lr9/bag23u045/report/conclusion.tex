\ssr{ЗАКЛЮЧЕНИЕ}

В заключение, проведенный анализ продемонстрировал применение различных методов кластеризации — K-Means, C-Means и иерархической кластеризации — к исследуемому набору данных. 
Использование метрик внутрикластерного и межкластерного расстояний позволило оценить влияние количества кластеров на компактность и разделение групп данных. 
Графики, демонстрирующие динамику этих расстояний, а также визуализации самих кластеров, выявили общую тенденцию: увеличение количества кластеров приводит к уменьшению 
внутрикластерного расстояния, но в определенный момент это уменьшение замедляется, а межкластерное расстояние, достигнув некоторого значения, перестает существенно расти.

Анализ визуализаций кластеров показал, что все три метода кластеризации в целом сформировали похожие кластеры, несмотря на различия в их алгоритмах. 
Однако, C-Means, как метод нечеткой кластеризации, продемонстрировал большую гибкость в случаях, когда данные имели некоторую степень пересечения. 
Иерархическая кластеризация предоставила более полное представление о структуре кластеров, позволяя оценить иерархию их объединения. 
K-Means также показал себя достаточно эффективно, хотя и с некоторой потерей гибкости в сравнении с C-Means.

Оценка оптимального количества кластеров, проведенная как с помощью "метода локтя", так и визуального анализа, привела к заключению о том, 
что разбиение данных на 5 кластеров является наиболее адекватным, что отличается от оценки эксперта в 7 кластеров. 
Этот выбор позволяет достичь баланса между компактностью кластеров и их разделением. 

В целом, полученные результаты подчеркивают эффективность всех примененных методов для кластеризации данных и указывают на то, 
что при правильной интерпретации и оценке их результатов, они могут быть успешно применены для выявления скрытых закономерностей в данных.