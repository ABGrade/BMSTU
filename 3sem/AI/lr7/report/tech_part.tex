\chapter{Технологическая часть}

\section{Средства написания программы}

Среда разработки: PyCharm 2023 community edition

Язык разработки: python 3.10
	
Используемые библиотеки:
\begin{itemize}
    \item 
    os - для доступа к файлам системы и создания папок/файлов
    URL: https://docs-python.ru/standart-library/modul-os-python/
    \item 
    numpy - для удобной и быстрой работы с массивами, сохранения промежуточных данных и связи их с другими библиотеками
    URL: https://numpy.org/doc/
    \item 
    re - для выделения слов
    URL: https://docs-python.ru/standart-library/modul-re-python/
    \item 
    collections.defaultdict - для создания начальных словарей в формате (строка : число)
    URL: https://docs-python.ru/standart-library/modul-collections-python/klass-defaultdict-modulja-collections/
    \item 
    bs4.BeautifulSoup - для чтения html файлов
    URL: https://www.crummy.com/software/BeautifulSoup/bs4/doc.ru/bs4ru.html
    \item 
    docx.Document - для чтения docx файлов
    URL: https://docs-python.ru/packages/modul-python-docx-python/klass-document/
    \item 
    odf.opendocument.load и odf.text.P - для чтения odt файлов
    URL: https://pypi.org/project/odfpy/
    \item 
    pymorphy3 - для выделения начальных форм слова
    URL: https://pypi.org/project/pymorphy3/
    \item 
    matplotlib.pyplot - для отображения графиков
    URL: https://matplotlib.org/stable/index.html
    \item 
    seaborn - для построения графиков
    URL: https://seaborn.pydata.org/
    \item 
    sklearn.metrics.pairwise.cosine similarity - косинусная мера близости
    URL: https://scikit-learn.org/stable/
\end{itemize}

\section{Подготовка к анализу}

Основные функции программы preproc.py, которая подготавливает данные для их дальнейшего анализа.

\subsection{Построение векторов}

Функция создания вектора

\begin{lstlisting}[label=b-vector, caption={Построение векторов (обычного и нормализированного для одного документа)}]
    def build_vectors(text):
        morph = pymorphy3.MorphAnalyzer()
        words = list(word.lower() for word in re.split(r"[ -.,!?():;\n\t\r]+", text) if word)
        word_count = defaultdict(int)
        word_count_norm = defaultdict(int)
        for word in words:
            word_count[word] += 1
            word_count_norm[morph.normal_forms(word)[0]] += 1
        return word_count, word_count_norm
\end{lstlisting}

\subsection{Рекурсивное чтение файлов, создания векторов и запись в файл}

Основная функция preproc.py, выполняющая все необходимые действия (чтение, создание, запись) для получения готовых к анализу векторов.

\begin{lstlisting}[label=b-vector-all, caption={Получение всех векторов}]
    def build_vectors_recursively(directory):
        for folder in os.listdir(directory):
            folder_path = os.path.join(directory, folder)
            vectors_dir_path = os.path.join(folder_path, "vectors")
            norm_vectors_dir_path = os.path.join(folder_path, "norm_vectors")

            if not os.path.exists(vectors_dir_path):
                os.mkdir(vectors_dir_path)

            if not os.path.exists(norm_vectors_dir_path):
                os.mkdir(norm_vectors_dir_path)

            for filename in os.listdir(folder_path):
                file_path = os.path.join(folder_path, filename)
                if file_path.endswith(EXT):
                    text = ext_choice(file_path)
                    if text is None:
                        print(f"None pointer return for file: {file_path}\n")
                        continue
                    vec, vec_norm = build_vectors(text)
                    write_into_file(vec, vectors_dir_path, filename)
                    write_into_file(vec_norm, norm_vectors_dir_path, filename)
\end{lstlisting}

\section{Метрики}

Метод Жаккара и косинусная мера близости.

\subsection{Метод Жаккара}

Ранее уже рассмотренный метод Жаккара: См. \ref{jac_sim}

\begin{lstlisting}[label=jac-func, caption={Метод Жаккара}]
    def jacquard_similarity(v1: dict, v2: dict):
        a = set(v1.keys())
        b = set(v2.keys())
        shared = a.intersection(b)
        total = a.union(b)
        return len(shared) / len(total)
\end{lstlisting}

\subsection{Косинусная мера близости}

Ранее уже рассмотренная мера близости: См. \ref{cos_sim}

\begin{lstlisting}[label=cos-func, caption={Косинусная мера близости}]
    def cosine_metric(v1: dict, v2: dict):
        keys = sorted(set(v1.keys()).union(v2.keys()))
        vector1 = np.array([v1.get(key, 0) for key in keys])
        vector2 = np.array([v2.get(key, 0) for key in keys])
        return cosine_similarity([vector1], [vector2])[0][0]
\end{lstlisting}

\section{Анализ}

\subsection{Матрица близости}

Матрица близости - это матрица, которая в каждой ячейке (i, j) содержит сравнительный анализ двух векторов.

\begin{lstlisting}[label=cor_matr, caption={Получение матрицы близости}]
    def get_corr_matrix(vectors):
        vecs_len = len(vectors)
        na = np.zeros((vecs_len, vecs_len))
        nb = np.zeros((vecs_len, vecs_len))
        for i in range(vecs_len):
            for j in range(vecs_len):
                na[i][j] = metrics.jacquard_similarity(vectors[i], vectors[j])
                nb[i][j] = metrics.cosine_metric(vectors[i], vectors[j])
        return na, nb
\end{lstlisting}

\subsection{Карта сравнения}

На основе матрицы, полученной из предыдущей функции, строится график в seaborn. Данный график является результатом всей работы.

\begin{lstlisting}[label=heatmaps, caption={Получение результирующих графиков}]
    def build_heatmaps_recursive(directory):
        vectors = list()
        norm_vectors = list()
        files = list()
        for folder in os.listdir(directory):
            folder_path = os.path.join(directory, folder)
            vectors_dir_path = os.path.join(folder_path, "vectors")
            norm_vectors_dir_path = os.path.join(folder_path, "norm_vectors")
            for filename in os.listdir(vectors_dir_path):
                files.append(filename)
                file_path = os.path.join(vectors_dir_path, filename)
                file_path_norm = os.path.join(norm_vectors_dir_path, filename)
                vectors.append(get_vectors(file_path))
                norm_vectors.append(get_vectors(file_path_norm))

        matrix_jacquard, matrix_cosine = get_corr_matrix(vectors)
        plt.figure(figsize=(19.2, 10.8))
        sea.heatmap(matrix_jacquard, xticklabels=files, yticklabels=files, square=True, linewidths=.3, cmap="YlGnBu")
        plt.savefig(os.path.join(RES_DIR, "Heatmap Jac.png"), dpi=100)
        plt.clf()

        plt.figure(figsize=(19.2, 10.8))
        sea.heatmap(matrix_cosine, xticklabels=files, yticklabels=files, square=True, linewidths=.3, cmap="YlGnBu")
        plt.savefig(os.path.join(RES_DIR, "Heatmap cos.png"), dpi=100)
        plt.clf()

        plt.figure(figsize=(19.2, 10.8))
        matrix_jacquard, matrix_cosine = get_corr_matrix(norm_vectors)
        sea.heatmap(matrix_jacquard, xticklabels=files, yticklabels=files, square=True, linewidths=.3, cmap="YlGnBu")
        plt.savefig(os.path.join(RES_DIR, "Heatmap Jac norm.png"), dpi=100)
        plt.clf()

        plt.figure(figsize=(19.2, 10.8))
        sea.heatmap(matrix_cosine, xticklabels=files, yticklabels=files, square=True, linewidths=.3, cmap="YlGnBu")
        plt.savefig(os.path.join(RES_DIR, "Heatmap cos norm.png"), dpi=100)
\end{lstlisting}

\section*{Вывод}

В результате получаем 4 графика (на каждую метрику отводится нормализированный и не нормализированный вектор). 
Эти графики в явном виде отображают близость каждого файла.

\clearpage
