\chapter{Технологическая часть}

\section{Средства написания программы}
Для реализации программного обеспечения были использованы следующие средства:

\begin{itemize}
    \item cреда разработки PyCharm 2023 community edition \cite{lib:pycharm}
    \item язык разработки Python 3.10 \cite{lib:python}
\end{itemize}
	
Используемые библиотеки:
\begin{itemize}
    \item pathlib — для доступа к файлам системы и создания папок/файлов \cite{lib:pathlib}
    \item numpy — для быстрой работы с массивами, сохранения промежуточных данных и связи их с другими библиотеками \cite{lib:numpy}
    \item cv2 — для работы с изображениями \cite{lib:opencv}
\end{itemize}

\section{Функции}

\begin{lstlisting}[caption={Определение ромба}]
    def detect_diamonds(image):
        gray = cv2.cvtColor(image, cv2.COLOR_BGR2GRAY)
        thresh = cv2.adaptiveThreshold(gray, 255, cv2.ADAPTIVE_THRESH_GAUSSIAN_C, cv2.THRESH_BINARY_INV, 11, 2)
        contours, _ = cv2.findContours(thresh, cv2.RETR_EXTERNAL, cv2.CHAIN_APPROX_SIMPLE)
        diamonds_found = False
        for contour in contours:
            epsilon = 0.04 * cv2.arcLength(contour, True)
            approx = cv2.approxPolyDP(contour, epsilon, True)
            if len(approx) == 4 and cv2.isContourConvex(approx):
                sides = [np.linalg.norm(approx[i][0] - approx[(i+1) % 4][0]) for i in range(4)]
                side_ratios = [sides[i] / sides[(i+1) % 4] for i in range(4)]
                if all(0.6 <= ratio <= 1.4 for ratio in side_ratios):
                    angles = []
                    for i in range(4):
                        vec1 = approx[i][0] - approx[(i-1) % 4][0]
                        vec2 = approx[i][0] - approx[(i+1) % 4][0]
                        cos_angle = np.dot(vec1, vec2) / (np.linalg.norm(vec1) * np.linalg.norm(vec2))
                        angle = np.arccos(np.clip(cos_angle, -1.0, 1.0)) * (180.0 / np.pi)
                        angles.append(angle)
                    if all(angle > 25 and angle < 165 for angle in angles):
                        acute_angles = sum(1 for angle in angles if 45 <= angle <= 85)
                        obtuse_angles = sum(1 for angle in angles if 95 <= angle <= 135)
                        if acute_angles == 2 and obtuse_angles == 2:
                            diag1_sq = np.sum((approx[0][0] - approx[2][0])**2)
                            diag2_sq = np.sum((approx[1][0] - approx[3][0])**2)
                            if not 0.9 <= diag1_sq / diag2_sq <= 1.2:
                                x, y, w, h = cv2.boundingRect(approx)
                                cv2.rectangle(image, (x, y), (x + w, y + h), (255, 0, 0), 2)
                                diamonds_found = True
        if diamonds_found:
            return image
        else:
            return None
\end{lstlisting}

\begin{lstlisting}[caption={Определение лошади}]
    def detect_horse(image, model_path, horse_class_id):
        net = cv2.dnn.readNet(model_path, model_path.replace(".weights", ".cfg"))
        blob = cv2.dnn.blobFromImage(image, scalefactor=1/255.0, size=(416, 416), swapRB=True, crop=False)
        net.setInput(blob)
        layer_names = net.getUnconnectedOutLayersNames()
        outputs = net.forward(layer_names)
        h, w = image.shape[:2]
        boxes = []
        confidences = []
        for output in outputs:
            for detection in output:
                scores = detection[5:]
                class_id = np.argmax(scores)
                confidence = scores[class_id]
                if class_id == horse_class_id and confidence > 0.2:
                    center_x, center_y, width, height = detection[:4] * np.array([w, h, w, h])
                    x = int(center_x  width / 2)
                    y = int(center_y  height / 2)
                    boxes.append([x, y, int(width), int(height)])
                    confidences.append(confidence)
        if len(boxes) == 0:
            return None
        indices = cv2.dnn.NMSBoxes(boxes, confidences, score_threshold=0.3, nms_threshold=0.7)
        if len(indices) > 0:
            for i in indices.flatten():
                box = boxes[i]
                x, y, w, h = box
                cv2.rectangle(image, (x, y), (x + w, y + h), (24, 143, 0), 2)
                cv2.putText(image, f'{class_names[horse_class_id]}: {confidences[i] * 100:.2f}%',
                            (x + 10, y - 10), cv2.FONT_HERSHEY_DUPLEX, 1, (0, 255, 0), 1)
        return image
\end{lstlisting}

\begin{lstlisting}[caption={Обработка изображения}]
    def proceed_image(image_path, model_path):
        horse_class_id = class_names.index("horse")
        image = cv2.imread(image_path)
        image_copy = image.copy()
        diamond = detect_diamonds(image)
        image_with_boxes = detect_horse(image_copy, model_path, horse_class_id)
        if image_with_boxes is not None:
            output = RES_DIR / "horses" / image_path.name
            output.parent.mkdir(parents=True, exist_ok=True)
            cv2.imwrite(str(output), image_with_boxes)
        if diamond is not None:
            output = RES_DIR / "diamonds" / image_path.name
            output.parent.mkdir(parents=True, exist_ok=True)
            cv2.imwrite(str(output), diamond)
\end{lstlisting}

\section*{Вывод}

Перечисленные технологии и функции позволяют реализовать программу для обнаружения ромбов и лошадей в изображениях.

\clearpage
