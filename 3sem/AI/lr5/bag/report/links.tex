\addcontentsline{toc}{chapter}{СПИСОК ИСПОЛЬЗОВАННЫХ ИСТОЧНИКОВ}
\begin{thebibliography}{}
	% \bibitem{lib:os}
	% Модуль os в Python, доступ к функциям ОС
    % URL: https://docs-python.ru/standart-library/modul-os-python/  (Дата обращения 07.01.2025)
	\bibitem{lib:numpy}
	NumPy Documentation
    URL: https://numpy.org/doc/ (Дата обращения 07.01.2025)
	\bibitem{lib:matplotlib}
	Matplotlib 3.9.2 documentation
    URL: https://matplotlib.org/stable/index.html (Дата обращения 07.01.2025)
	% \bibitem{lib:seaborn}
	% seaborn: statistical data visualization
    % URL: https://seaborn.pydata.org/ (Дата обращения 07.01.2025)
	\bibitem{lib:sklearn}
	Документация sklearn
    URL: https://scikit-learn.org/stable/ (Дата обращения 07.01.2025)
	\bibitem{lib:tensorflow}
	Документация Tensorflow
	URL: https://www.tensorflow.org/api\_docs (Дата обращения 07.01.2025)
	% \bibitem{lib:pandas}
	% Документация Pandas
	% https://pandas.pydata.org/pandas-docs/stable/index.html (Дата обращения 07.01.2025)
	\bibitem{lib:pathlib}
	Документация Pathlib
	https://docs.python.org/3/library/pathlib.html (Дата обращения 07.01.2025)
	\bibitem{lib:pycharm}
	Среда разработки Pycharm 2023 Community edition
	URL: https://www.jetbrains.com/ru-ru/pycharm/download/other.html
	\bibitem{lib:python}
	Язык разработки Python 3.10
	URL: https://www.python.org/downloads/release/python-3100/
	\bibitem{lib:windows}
	ОС Windows 11
	URL: https://www.microsoft.com/ru-ru/software-download/windows11
	\bibitem{lib:intel}
	Intel i5-12500H
	URL: https://www.intel.com/content/www/us/en/products/sku/96141/intel-core-i512500h-processor-18m-cache-up-to-4-50-ghz/specifications.html?wapkw=12500h
	\bibitem{lib:neuron}
	Введение в архитектуру нейронных сетей
	URL: https://cyberleninka.ru/article/n/vvedenie-v-arhitekturu-neyronnyh-setey (Дата обращения 07.01.2025)
	\bibitem{lib:ReLU}
	Функия активации
	URL: https://cyberleninka.ru/article/n/sravnenie-sposobov-vychisleniya-proizvodnoy-aktivatsii-vypryamitel-pri-obuchenii-neyronnoy-seti (Дата обращения 07.01.2025)
	\bibitem{lib:KLD}
	Функция потерь
	URL: https://cyberleninka.ru/article/n/why-deep-learning-methods-use-kl-divergence-instead-of-least-squares-a-possible-pedagogical-explanation (Дата обращения 07.01.2025)
	\bibitem{lib:softmax}
	Функция активации softmax
	URL: https://cyberleninka.ru/article/n/ispolzovanie-modeli-neyronnoy-seti-glubokogo-obucheniya-dlya-resheniya-problem-klassifikatsii-nezhelatelnogo-kontenta-v-sotsialnyh (Дата обращения 07.01.2025)
	\bibitem{lib:chebyshev}
	Неравенство Чебышева
	URL: https://cyberleninka.ru/article/n/k-obosnovaniyu-metoda-ustoychivogo-otsenivaniya-posredstvom-neravenstva-chebyshyova (Дата обращения 07.01.2025)
\end{thebibliography}