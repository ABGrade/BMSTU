\chapter{Аналитическая часть}

Компьютерное зрение представляет собой одну из ключевых областей искусственного интеллекта, 
направленную на автоматизацию задач, выполняемых человеческим зрением, таких как распознавание объектов, 
определение их местоположения, анализ движения и интерпретация сцены. 
Технологии компьютерного зрения находят широкое применение в различных сферах, 
включая автономные системы, медицинскую диагностику, промышленные системы контроля и робототехнику. 
Одной из фундаментальных задач этой области является определение объектов на изображении.

\section{Определение объектов на изображении}

Определение объекта на изображении в общем случае включает несколько этапов. 
Сначала выполняется предобработка данных, включающая устранение шумов, 
нормализацию изображения и его масштабирование для дальнейшего анализа. 
Затем происходит извлечение признаков, которое может быть основано на классических методах анализа изображения, 
таких как выделение краев, или на применении нейронных сетей, которые автоматически выделяют характерные черты объектов. 
Следующим этапом является классификация, где модель определяет принадлежность объекта к заданному классу, 
а также локализация, направленная на определение координат объекта в пространстве изображения. 
Для устранения ложных срабатываний и улучшения точности результатов на завершающем этапе применяется постобработка.

\section{Технологии компьютерного зрения}

OpenCV (Open Source Computer Vision Library) является широко применяемым инструментом для работы с изображениями и видео. 
Она предоставляет обширный набор функций для выполнения таких задач, как обработка изображений, анализ движения и распознавание объектов. 
Библиотека поддерживает интеграцию с современными технологиями машинного обучения, включая TensorFlow и PyTorch, 
что делает её подходящим выбором для реализации систем компьютерного зрения.

YOLO (You Only Look Once) представляет собой семейство моделей глубокого обучения, 
которые позволяют выполнять детектирование и классификацию объектов за один проход через нейронную сеть. 
Это достигается за счёт разделения изображения на сетку и одновременного предсказания классов и координат объектов. 
Высокая скорость работы делает YOLO особенно эффективным для задач реального времени.

Алгоритм NMS (Non-Maximum Suppression) используется для удаления избыточных ограничивающих прямоугольников, 
предсказанных для одного и того же объекта. 
NMS основывается на вычислении метрики IoU (Intersection over Union), которая определяет степень перекрытия прямоугольников. 
Прямоугольники с высокой степенью перекрытия, кроме одного с наивысшей вероятностью, удаляются, что позволяет значительно улучшить точность детектирования.

\section*{Вывод}

В ходе аналитического исследования были рассмотрены ключевые аспекты компьютерного зрения, 
включая процессы детектирования объектов, а также основные технологии, такие как библиотека OpenCV, алгоритмы YOLO и NMS. 
Данные инструменты предоставляют широкие возможности для реализации задач автоматического анализа изображений, включая поиск заданных объектов. 

\clearpage

